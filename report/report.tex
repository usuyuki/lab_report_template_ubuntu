\documentclass[mingoth]{jsarticle}
% 最初の段落も字下げする
\usepackage{indentfirst}
% デフォルトだとtabみたいな字下げするので1文字だけにする
\setlength{\parindent}{1em}
% Figとtableの日本語化
\def\fnum@figure{図}
\def\fnum@table{表}
% リスト(ソースコード)はjlistingで日本語になる
% 画像関連                                     
\usepackage[dvipdfmx]{graphicx}
% 表関連
\usepackage{amssymb, amsmath}
\usepackage{booktabs}
% コードブロック関連 jlistingにより日本語コメントアウトでも崩れないようにできる
\usepackage{listings,jlisting,here}
\lstset{
	basicstyle={\ttfamily},
	identifierstyle={\small},
	commentstyle={\smallitshape},
	keywordstyle={\small\bfseries},
	ndkeywordstyle={\small},
	stringstyle={\small\ttfamily},
	frame={tb},
	breaklines=true,
	columns=[l]{fullflexible},
	numbers=left,
	xrightmargin=0zw,
	xleftmargin=3zw,
	numberstyle={\scriptsize},
	stepnumber=1,
	numbersep=1zw,
	lineskip=-0.5ex
}

\title{レポートタイトル}
\author{著者名}
\date{\today}

\begin{document}

\maketitle

\section{概要}

\section{タイトル}
プログラムのソースコードをリスト\ref{blockCode}に示す.
\lstinputlisting[caption=リストタイトル,label=blockCode]{../src/block_matmul.c}

プログラムを実行したときの様子を図\ref{block_sc}に示す.
\begin{figure}[H]
	\centering
	\includegraphics[keepaspectratio,width=1.0\linewidth]{img/block_sc.png}
	\caption{行画面}
	\label{block_sc}
\end{figure}

\section{考察}


\section{感想}


\end{document}
